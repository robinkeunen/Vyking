% !TEX encoding = UTF-8
\documentclass[a4paper,11pt]{article}

\usepackage[utf8]{inputenc}
\usepackage[francais]{babel}
\usepackage[T1]{fontenc}


%math 
\usepackage{amsmath}
%\usepackage{amssymb} // Decommenter si le dernier n'est pas suffisant

%Chimie
\usepackage[version=3]{mhchem}
%cf http://fr.wikibooks.org/wiki/LaTeX/%C3%89crire_des_formules_chimiques

% Allows for temporary adjustment of side margins
% Lorsque les tableaux sont trop grands
\usepackage{chngpage}
 \usepackage{multirow} 
 
% provides filler text
\usepackage{lipsum}
% usage : \lipsum[1]

% just makes the table prettier (see \toprule, \bottomrule, etc. commands below)
\usepackage{booktabs}

%Mise en page
\usepackage{graphicx}
\usepackage{fancyhdr}
\usepackage[top=3cm, bottom=3cm, left=4cm, right=3cm]{geometry}
\usepackage{listings}
\usepackage{xcolor}
\usepackage{placeins}

\definecolor{gris}{gray}{0.5}
\definecolor{code}{gray}{0.95}

% Table des matières cliquable 
\usepackage{hyperref}
\hypersetup{
    colorlinks, % empécher latex de colorer les liens
    citecolor=black,
    filecolor=black,
    linkcolor=black, % couleur des liens dans la table des matières
    urlcolor=blue
}

%Points dans la table des matières
\usepackage{tocloft}
\renewcommand{\cftsecleader}{\cftdotfill{\cftdotsep}} % Ligne de points dans la table des matières


%Page de garde

\newlength{\larg}
\setlength{\larg}{14.5cm}

% Pour corriger alignement, jouer avec la taille de tabular (± 6cm)
\title{
{\rule{\larg}{1mm}}\vspace{7mm}
\begin{tabular}{p{5,2cm} r}
 \multirow{3}{*}{\includegraphics[width=70px]{vyking-logo2.jpg}} & {\Huge {\bf Langage Vyking}} \\
   & \\
   & {\Large Grammaire et spécifications du language}
\end{tabular}\\
\vspace{2mm}
{\rule{\larg}{1mm}}
\vspace{2mm} \\
\begin{tabular}{p{9.5cm} r}
   & {\large \bf Compilateurs} \\
   & {\large \bf INFO0085} \\
   &{\large \bf \bsc{Pr. Pierre Geurts}}\\
   &{\large \bf \bsc{Cyril Soldani}}\\
   & {\large  \today}
\end{tabular}\\
\vspace{10cm}
}
\author{\begin{tabular}{p{13.7cm}}
\bsc{Robin Keunen} s093137\\
\bsc{Pierre Vyncke} s091918\\
1\up{ème} Master Ingénieur Civil
\end{tabular}\\
\hline }
\date{}

\usepackage{times}
\newcommand{\be}{\begin{enumerate}}
\newcommand{\ee}{\end{enumerate}}
\newcommand{\bi}{\begin{itemize}}
\newcommand{\ei}{\end{itemize}}

\definecolor{quotationcolour}{HTML}{F0F0F0}
\definecolor{quotationmarkcolour}{HTML}{1F3F81}

% Double-line for start and end of epigraph.
\newcommand{\epiline}{\hrule \vskip -.2em \hrule}
% Massively humongous opening quotation mark.
\newcommand{\hugequote}{%
  \fontsize{42}{48}\selectfont \color{quotationmarkcolour} \textbf{``}
  \vskip -.5em
}

% Beautify quotations.
\newcommand{\epigraph}[2]{%
  \bigskip
  \begin{flushright}
  \colorbox{quotationcolour}{%
    \parbox{.60\textwidth}{%
    \epiline \vskip 1em {\hugequote} \vskip -.5em
    \parindent 2.2em
    #1\begin{flushright}\textsc{#2}\end{flushright}
    \epiline
    }
  }
  \end{flushright}
  \bigskip
}

\begin{document}

% Titre
\maketitle
\thispagestyle{empty}
\newpage

% Mise en page
\pagestyle{fancy}
\lhead{}
\chead{}
\rhead{\itshape \textcolor{gris}{Compilateurs}}
\lfoot{\itshape \textcolor{gris}{Grammaire et spécifications du language}}
\cfoot{}
\rfoot{\itshape \textcolor{gris}{\thepage}}
\renewcommand{\headrulewidth}{0.4pt}
\renewcommand{\footrulewidth}{0.4pt}

\newpage 

\section{Introduction}
	Le but de Vyking est d'étudier l'implémentation de langages de programmation fonctionnels.
	Nous nous inspirerons principalement de Scheme et de Python.
	Nous avons apprécié la puissance de Scheme et de l'usage des listes.
	Les listes seront donc une structure de base de notre langage.
	Nous apprécions aussi la concision et la syntaxe claire de Python.
	
	Le nom du langage vient de Vy (Vyncke) - (Keunen) -ing. 
		
\section{Spécifications}

	\subsection*{Syntaxe}
		Nous voulons que Vyking soit un langage de haut niveau.
		le code doit être lisible, concis et élégant. 
		Nous sommes particulièrement sensible au "PEP 20 (The Zen of Python)" : 
		\bi 
		\item Beautiful is better than ugly.
		\item Explicit is better than implicit.
		\item Simple is better than complex.
		\item Complex is better than complicated.
		\item Readability counts.
		\ei 
		Le langage sera donc épuré d'accolades, de points-virgules et autre ponctuation inutile.
		La tabulation délimitera les blocs.
		Outre le look épuré du code, ce choix devrait forcer les utilisateurs du langages à se conformer à un seul style d'indentation (contrairement à C et ses styles K\&R, Allman, Whitesmiths, ...).
	
	\subsection*{Listes}
		Nous avons trouvé que l'utilisation des listes dans Scheme était très élégante pour implémenter certain schémas de récursion.
		La structure de liste sera donc implémentée directement dans le langage.
		Les listes seront construites à partir de paires pointées.
		
		Vyking fournira plusieurs fonctions de bases pour manipuler les listes : \verb cons ,  \verb append , \verb list,  \verb map  et  \verb apply .
		Nous remplaçons les \verb car  et \verb cdr  de Scheme qui ne sont pas très intuitif par \verb head  et \verb tail .
		
		Si nous en avons le temps, nous tenterons d'implémenter des "list comprehension".
	
	\subsection*{Fonctions de première classe}
		Pour rendre le langage réellement fonctionnel, il semble indispensable de pouvoir manipuler les fonctions comme des objets de première classe.
		Nous voulons pouvoir passer les fonctions en argument, les retourner à partir d'autres fonctions et permettre les définitions de fonctions imbriquées.		
		Pour pouvoir supporter les fonctions imbriquées, nous avons besoin de lier l'environnement à la définition d'une fonction.
		Nous tenterons donc d'implémenter des "closures" (fermeture).
		
		L'implémentation des closures semble compliquée.
		Dans un premier temps, nous nous contenterons de passer les fonctions en argument par des pointeurs de fonctions.
		Les closures viendront dans un second temps.

\section{Choix du langage pour l'implémentation}
	Le compilateur sera implémenté en Python.
	La plupart des projets pour les cours sont à rendre en Java ou en C/C++.
	Scheme (ou Racket) semble adapté pour l'analyse du langage mais il est peu utilisé dans des projets hors du secteur académique.
	Nous voulions nous former à un autre langage, plus moderne.
	Nous avons choisis Python, l'expressivité du langage et la richesse des librairies devrait nous permettre de nous concentrer sur les algorithmes et sur Vyking.

\section{Grammaire}
	\begin{verbatim}
	(* language components *)
file_input = {NEWLINE | statement}, ENDMARKER

keyword = 'return' | 'defun' 
        | 'and' | 'or' | 'not' | 'is' 
        | 'if' | 'elif' | 'else'
        | 'while' | 'for'
        | 'map' | 'apply' | 'cons' | 'append' | 'list' 
        | 'head' | 'tail | 'in';
        
statement = (funcall
                    | if_stat
                    | while_stat
                    | for_stat
                    | let_stat
                    | return_stat
                    | funcdef_stat
                    | import_stat
                    ), NEWLINE;
     
funcall = id, '(', [args], ')' | list_fun;
fundef = 'defun', id, parameters, ':', block;
if_stat = 'if', test, ':', block, {'elif' test ':' block}, ['else' ':' block];
while_stat = 'while', test, ':', block;
for_stat = 'for', id, 'in', list, ':', block;
let_stat = id, '=', (object | exp);
return_stat = 'return', (exp | list);
import_stat = 'from', id, 'import', name;

list_fun = cons_fun | append_fun | list_fun | head_fun | tail_fun | map_fun;
cons_fun = 'cons', '(', list, ')' | atom, '+', list;
append_fun = 'append', '(', list, ')' | list, '+', list;
list_fun = 'list', '(', list, ')';
head_fun = 'head', '(', list, ')';
tail_fun = 'tail', '(', list, ')';
map_fun = 'map', '(', id, ',', list, ')';

exp = add_exp;
add_exp = mul_exp, {('+' | '-'), mul_exp};
mul_exp = atom {('*' | '/' | '%'), atom}
atom = id | int | float | 'None' | bool | funcall;

block = statement | NEWLINE INDENT statement {statement} DEDENT;
test = xor_test
xor_test = or_test, {'|', or_test};
or_test = and_test, {'or', and_test};
and_test = not_test {'and', not_test};
not_test = 'not' not_test | identity | atom;
identity = comparison, ['==', atom]
comparison = exp, [comp_op, exp];

parameters = '(', [ids], ')';
ids = id, ',', ids | id;
args = arg, ',', args | arg;
arg = id | funcall | exp;

(* base types *)
int = [(+|-)], digit_0, {digit} | '0';
float = int, '.', [digit, {digit}] [exponent];
exponent = ('e' | 'E'), int;
char = (letter | digit | ' ');
esc_char = '\', letter;
bool = 'True' | 'False';

(* complex types *)
string = '"', {char | esc_char |"'" }, '"' | "'", {char | '"'}, "'";
list = '[', {object, ','}, 'object', ']' | '[', ']' | list_fun;

object = int | float | char | bool | string | list | id;

(* base cases *)
id = (letter | '_'), { letter | digit | '_'} -keyword;
letter = uppercase | lowercase;
uppercase = 'A' | 'B' | 'C' | 'D' | 'E' | 'F' | 'G'
          | 'H' | 'I' | 'J' | 'K' | 'L' | 'M' | 'N'
          | 'O' | 'P' | 'Q' | 'R' | 'S' | 'T' | 'U'
          | 'V' | 'W' | 'X' | 'Y' | 'Z' ;
lowercase = 'a' | 'b' | 'c' | 'd' | 'e' | 'f' | 'g'
          | 'h' | 'i' | 'j' | 'k' | 'l' | 'm' | 'n'
          | 'o' | 'p' | 'q' | 'r' | 's' | 't' | 'u'
          | 'v' | 'w' | 'x' | 'y' | 'z' ;
digit = '0' | '1' | '2' | '3' | '4' | '5' | '6' | '7'
      | '8' | '9';
digit_0 = '1' | '2' | '3' | '4' | '5' | '6' | '7' | 
        | '8' | '9';
tab = '    ' | '\t';
comp_op = '==' | '<' | '>' | '<=' | '>=' | '!=' | 'is';

	\end{verbatim}


\end{document}	





















